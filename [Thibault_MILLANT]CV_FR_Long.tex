\documentclass[11pt,a4paper]{moderncv}
\moderncvtheme[blue]{classic} % Thèmes existant : oldstyle, classic, casual, banking; Couleurs du thème existant : blue, orange, green, red, purple, grey, black
\usepackage[utf8]{inputenc}
\usepackage{geometry}
\geometry{hscale=0.80,vscale=0.80,centering}

\firstname{Thibault}
\familyname{MILLANT}
\title{02/09/1992 - Étudiant en école d'ingénieur}
\address{34 Boulevard Cointet}{06400 Cannes}
\email{tmillant@gmail.com}
\mobile{+1 617 337 7116}
\phone{+33 4 93 48 88 46}
\extrainfo{Twitter @MThibault06, Permis B, A1}
\photo{photo.jpg}

\begin{document}
\maketitle
\section{Formations} % Partie Formations
\cventry{2015-Auj}{École d'ingénieur}{Computer Science \& Computer Information System}{}{Los Angeles}{\textbf{\textit{California State University Los Angeles}}}
\cventry{2012-Auj}{École d'ingénieur}{Paris}{Majeure Systèmes d'Information}{Mineure Projet Personnel}{\textbf{\textit{ECE Paris}}} % ligne du diplôme
\cventry{Sept 2013 - Déc 2013}{Semestre d'Échange en Computing Science}{Montréal}{Canada}{}{\textbf{\textit{Université Concordia}}}
\cventry{2010-2011}{Faculté de Médecine}{Nice}{France}{}{\textbf{\textit{UFR Médecine Nice-Sophia Antipolis}}}
\cventry{1995-2010}{Baccalauréat Scientifique Européen Italien, Mention Assez Bien}{Cannes}{France}{}{\textbf{\textit{Institut Stanislas}}}

\section{Expériences} % partie Compétences
\cventry{Mai 2015 - Août 2015}{Stage Data Analyst \& Software Developer}{\textbf{\textit{Amadeus North America}}}{Boston}{USA}{
\begin{itemize}
  \item Stage au sein des divisions AIR (Availibility) et SSP (Search, Shopping and Pricing) en R\&D.
  \item Analyse d'un important flux de données (Big Data) afin d'améliorer l'efficacité des solutions proposées par l'entreprise.
  \item Réalisation d'un programme fournissant des graphiques et des statistiques pour automatiser les analyses.\\
\end{itemize}}
\cventry{Jan 2014-Auj}{Assistant Professeur TP Informatique}{\textbf{\textit{ECE Paris}}}{}{}{
\begin{itemize}
  \item Assister le professeur lors des séances de TPs et de TDs.
  \item Répondre aux questions des élèves.
  \item Débloquer les élèves et les assister dans leur rédaction de code informatique.
  \item Aider les élèves à acquérir de bonnes pratiques en terme d'algorithmies et de programmation.
  \item Corriger les examens.\\
\end{itemize}}
\cventry{Juillet 2011 - Aujourd'hui}{Professeur d'informatique}{\textbf{\textit{Freelance}}}{}{}{Cours d’informatique à domicile pour tous niveaux et tous OS, installations, dépannages, etc...
\begin{itemize}
  \item Apprentissage des bases de l'informatique.
  \item Accompagnement des clients dans leurs projets (montage vidéo, etc...).
  \item Conseils et accompagnement lors d'achats de matériel informatique.\\
\end{itemize}}
\cventry{Décembre 2014 - Aujourd'hui}{Grenelle du Numérique}{\textbf{\textit{ECE Paris \& CGI}}}{}{}{
\begin{itemize}
  \item Rédaction d'un livre blanc sur le sujet \textbf{Big data, ce mot qui va transformer les business models}.
  \item Prise de parole lors de conférence sur le Big Data (Paris Dauphine).
  \item Réalisation d'un spot vidéo informatif.
  \item Faire part de notre expertise aux députés avant le vote des lois sur le numérique de septembre 2015.\\
\end{itemize}}
\cventry{Octobre 2014 - Avril 2015}{Stage/Mineure Projet Personnel en Sécurité Informatique}{\textbf{\textit{Assemblée-Nationale}}}{}{}{
\begin{itemize}
  \item Étude de l'infrastructure de l'Assemblée-Nationale.
  \item Étude comparative avec le guide d'hygiène informatique de l'ANSSI et la PSSIE (Politique de Sécurité des Systèmes d'Information de l'État) dans le but d'améliorer la sécurité de l'infrastructure actuelle. Système de notation afin de prioriser les tâches et d'évaluer les risques pour le SI.
  \item Étude de la sécurité de l'infrastructure de l'Assemblée-Nationale dans le but d'obtenir un accès à une plate-forme sous mention \textbf{Diffusion Restreinte}.
  \item Rédaction de bulletins d'information sur les bonnes pratiques de sécurité informatique pour les députés et le personnel.
  \item Rédaction de guides complets de sécurisation de son smartphone (Android et IOS).
  \item Création et configuration d'un master Linux pour les anciens postes informatiques sortant de l'entreprise à destination d'organisations étatiques partenaires.
  \item Réalisation d'une procédure d'effacement sécurisé de données.\\
\end{itemize}}
\cventry{Juin 2014 - Août 2014}{Stage Développement Web et IOS}{\textbf{\textit{SCM Centre d'Imagerie Médicale Belvédère}}}{}{}{
\begin{itemize}
  \item Rédaction d'une proposition de cahier des charges pour répondre aux besoins de l'entreprise et être accepté en stage.
  \item Formation du personnel sur l'utilisation d'un outil existant (Outlook) afin de gérer un planning partagé.
  \item Création d'un site Web pour gérer les emplois du temps des médecins de la société.
  \item Auto-hébergement du site Web et intégration au sein de l'infrastructure de la société.
  \item Création d'une application IOS permettant la consultation des emplois de temps.
  \item Formation du personnel à l'utilisation de ces outils.
  \item Maintenance du serveur ajouté et interventions en cas de problèmes.\\
\end{itemize}}
\cventry{Juin 2013 - Juillet 2013}{Stage Sécurité Informatique et Pentest}{\textbf{\textit{Schneider-Electric}}}{Carros}{}{
\begin{itemize}
  \item Stage dans le domaine de la sécurité sur des automates.
  \item Mise en place de protocoles de test de sécurité.
  \item Recherche de vulnérabilités sur les automates, et exploitations.
  \item Rédaction de scripts de test de sécurité.
  \item Rédaction d'un rapport technique d'analyse de failles en anglais et en français.\\
\end{itemize}}
\cventry{Avril 2014 - Aujourd'hui}{Trésorier et Administrateur Système}{\textbf{\textit{iTeam}}}{}{}{
Association de promotion du logiciel libre et de sécurité informatique.
\begin{itemize}
  \item Gestion du budget de l'association.
  \item Organisations d’événements : conférences (GDG France, ...), formations (Virtualisation, Serveurs, Raspberry Pi, Sécurité informatique, \LaTeX, Git, ...), petit-déjeuner, etc...
  \item Voyage au FOSDEM.
  \item Gestion des serveurs de l'association et des services associés.\\
\end{itemize}}
\cventry{Sept. 2012 - Avril 2014}{Vice-Président Microsoft}{\textbf{\textit{Équipe Système}}}{}{}{
Association d'informatique de l'ECE Paris.
\begin{itemize}
  \item Poste dirigeant au sein de l'association.
  \item Responsable du pole Microsoft (organisation d'événement en lien avec le poste d'Ambassadeur Microsoft.
  \item Assistance des élèves, en coopération étroite avec le service informatique de l'école, afin de régler les soucis liés à l'infrastructure de l'école (comptes, imprimantes, Wi-Fi, etc...).\\
\end{itemize}}
\cventry{Sept. 2012 - Mai 2013}{Ambassadeur Microsoft}{\textbf{\textit{Microsoft France}}}{}{}{
\begin{itemize}
  \item Représentant Microsoft au sein de mon Campus.
  \item Formations et démonstrations des derniers produits Microsoft.\\
\end{itemize}}
\cventry{Nov. 2011 - Aujourd'hui}{Administrateur}{\textbf{\textit{Freedom-IP}}}{}{}{Administrateur d’un service de VPN communautaire et de son forum.
\begin{itemize}
  \item Gestion d'un forum et de ses utilisateurs.
  \item Gestion des comptes VPN des utilisateurs.
  \item Résolutions des problèmes des utilisateurs des services.\\
\end{itemize}}
%\cventry{Oct. 2013 - Dec. 2014}{Administrateur}{\textbf{\textit{Journal du Pirate}}}{}{}{
%\begin{itemize}
%  \item Rôle de gestion du forum.
%  \item Assistance et conseils envers les membres.
%  \item Entretien du forum.
%\end{itemize}}
%\cventry{Nov. 2011 - Oct. 2013}{Modérateur}{\textbf{\textit{Journal du Pirate}}}{}{}{
%\begin{itemize}
%  \item Rôle de gestion du forum.
%  \item Assistance et conseils envers les membres.
%  \item Entretien du forum.
%\end{itemize}}

\section{Compétences} % partie Compétences
\cvitem{}{\textbf{Programmation}}
\cvitem{Web}{HTML5, CSS3, PHP5, JS, Twitter bootstrap, CakePHP}
\cvitem{Logiciel}{C, C++, Java, Python}
\cvitem{Librairies Graphiques}{GTK, Allegro}
\cvitem{Bases de Données}{Microsoft Access, MySQL}
\cvitem{Autres}{Bash, \LaTeX}
\cvitem{Électronique}{VHDL} % liste d'items
\cvitem{}{}

\cvitem{}{\textbf{Systèmes d'exploitation}} % une double liste
\cvitem{}{Debian et ses dérivées (Ubuntu, KaliLinux, etc...), Archlinux, Windows (95 - 10), OSX}
\cvitem{}{}

\cvitem{}{\textbf{Logiciels}} % une double liste
\cvitem{Virtualisation}{VirtualBox, VMWare}
\cvitem{Bureautique}{LibreOffice, Microsoft Office 2003, 2007, 2010, 2013, 365} % une double liste
\cvitem{IDE}{Codeblocks, Eclipse, NetBeans, PHPStorm} % une double liste
\cvitem{Accès à distance}{SSH, VNC, Teamviewer} % une double liste
\cvitem{Électronique}{PSpice/Schematics, Xilinx} % une double liste
\cvitem{Mathématique}{MatLab} % liste d'items
\newpage

\section{Langues} % partie Compétences
\cventry{}{Français}{Langue maternelle}{}{}{}
\cventry{}{Anglais}{Intermédiaire}{TOEIC : 805}{IELTS : 7.5/9}{}
\cventry{}{Italien}{Intermédiaire}{}{}{}

\section{Centres d'intéret}
\cventry{}{Informatique}{Linux (Membre d'une association sur le logiciel libre et la sécurité informatique : iTeam), Réseau, Serveur, Sécurité, Cryptographie, sites de challenges (Position : 1470 / 39413)}{}{}{}
\cventry{}{Nouvelles Technologies}{Raspberry Pi, Systèmes d'information, VPN}{}{}{}
\cventry{}{Sports}{Snowboard/Ski (Compétition), Wakeboard, Surf, Tennis, Krav Maga, Muay Thaï}{}{}{}
\cventry{}{Musiques}{Piano et Guitare durant 8 ans}{}{}{}

\end{document}

\documentclass[10pt,a4paper]{moderncv}
\moderncvtheme[blue]{casual} % Thèmes existant : oldstyle, classic, casual, banking; Couleurs du thème existant : blue, orange, green, red, purple, grey, black
\usepackage[utf8]{inputenc}
\usepackage[top=0.5cm, bottom=1.8cm, left=0.9cm, right=0.9cm]{geometry}

\firstname{Thibault}
\familyname{MILLANT}
\title{02/09/1992 - Ingénieur Sécurité / Tests d'intrusion}
\address{4 Allée des Pruniers}{06800 Cagnes sur Mer}
\email{tmillant@gmail.com}
\mobile{+33 7 63 07 78 42}
%\phone{+33 4 93 48 88 46}
\extrainfo{Twitter @MThibault06}
\extrainfo{Permis B, A1}
\photo[50pt]{photo.png}

\begin{document}
\maketitle

\section{Diplôme} % Partie Formations
\cventry{2015}{École d'ingénieur}{Computer Science \& Computer Information System (Security oriented)}{}{Los Angeles}{\textbf{\textit{California State University Los Angeles}}}
\cventry{2012 - 2016}{École d'ingénieur}{Majeure Systèmes d'Information}{Mineure Projet Personnel}{Paris}{\textbf{\textit{ECE Paris}}} % ligne du diplôme
%\cventry{Sept 2013 - Déc 2013}{Semestre d'Échange en Computing Science}{Montréal}{Canada}{}{\textbf{\textit{Université Concordia}}}
%\cventry{2010-2011}{Faculté de Médecine}{Nice}{France}{}{\textbf{\textit{UFR Médecine Nice-Sophia Antipolis}}}
%\cventry{1995-2010}{Baccalauréat Scientifique Européen Italien, Mention Assez Bien}{Cannes}{France}{}{\textbf{\textit{Institut Stanislas}}}


\section{Certifications/Formations}
\cventry{}{CEH : Certified Ethical Hacker}{EC-Council}{}{}{}
\cventry{}{Usable Security}{University of Maryland}{Coursera}{}{}
\cventry{}{Software Security (with distinction)}{University of Maryland}{Coursera}{}{}
\cventry{}{Hardware Security (with distinction)}{University of Maryland}{Coursera}{}{}
\cventry{}{Cryptographie}{Cybrary.it}{}{}{}
\cventry{}{SQL Injection}{Cybrary.it}{}{}{}
%\cventry{}{Cybersecurity and Its Ten Domains}{University System of Georgia}{Coursera}{}{}
%\cventry{}{CompTIA Linux+}{Cybrary.it}{}{}{}
\cventry{}{Advanced Penetration Testing}{Cybrary.it}{}{}{}
%\cventry{}{Protégez-vous efficacement contre les failles web}{OpenClassroom}{}{}{}
%\cventry{}{PCI-DSS}{Cybrary.it}{}{}{}
\cventry{}{Web Application Penetration Testing}{Cybrary.it}{}{}{}
\cventry{}{Injection Flaws}{Cybrary.it}{}{}{}



\section{Publications}
\cventry{}{Grenelle du Numérique}{Big Data, ce mot qui va transformer les Business Models}{\emph{Sécurité du Big Data}}{}{}{}
\cventry{}{Stage Pentest AViSTO}{Pentest / Tests d'Intrusion}{\textit{Cas pratique} sur le site e-commerce \emph{Hackazon}}{}{}{}


\section{Expériences} % partie Compétences
\cventry{2018}{Scorpio Group}{}{Sécurité organisationnelle, \textit{Sécurité de l'infrastructure, du réseau}, \textit{et de l'utilisateur final}}{}{}
\cventry{2016}{Amadeus}{}{Kalilinux, \textit{Web Applications, Web Services, Projets de sécurité}, \textit{Test d'intrusion}}{}{}
\cventry{2016}{AViSTO}{}{Kalilinux, \textit{Sécurité Web} (Apache, PHP, AJAX), \textit{Test d'intrusion}}{}{}
\cventry{2015}{Amadeus North America}{}{\textit{Programmation sécurisée} (JS, PHP, Python, C++)}{}{}
\cventry{2014/15}{Assemblée Nationale}{}{\textit{PSSI}, ANSSI, \textit{Effacement sécurisé}, Prévention}{}{}
%\cventry{2014/15}{iTeam}{}{Formateur Sécurité / Administrateur Système Sécurité}{}{}
\cventry{2014}{SCM Centre d'Imagerie Médicale Belvédère}{}{\textit{Programmation Sécurisée} (PHP5,MySQL), \textit{Sécurité Système} (Raspberry, Apache), \textit{Sécurité Réseau}, Formateur}{}{}
\cventry{2013}{Schneider-Electric}{}{\textit{Test d'intrusion}, Automate, \textit{Kalilinux}, HTTP, FTP, Bash, Python}{}{}
%\cventry{2012/13}{Microsoft France}{}{Ambassadeur Microsoft / \textit{Formateur}}{}{}
%\cventry{2012/14}{Équipe Système}{}{Vice-Président Microsoft}{}{}


\section{Compétences} % partie Compétences
\cventry{\underline{Langages}}{Sécurité Applicative}{C/C++ (GTK), Python, Java, MySQL}{}{}{}
\cventry{}{Sécurité Web}{HTML, CSS, Javascript, PHP}{}{}{}
%\cventry{\underline{Outils}}{Backtrack/Kalilinux}{Virtualisation (VMWare, Virtualbox)}{}{}{}
\cventry{\underline{Réseaux}}{Protocoles}{HTTP, TCP/IP, SSL/TLS, SSH, FTP}{}{}{}
%\cventry{\underline{OS}}{Linux}{Windows, OSX, IOS, Androïd}{}{}{}


\section{Langues} % partie Compétences
\cvlistdoubleitem{\textbf{Anglais} : \textsl{Courant}, IELTS: 7.5/9}{\textbf{Italien} : \textsl{Intermédiaire}}


\section{Centres d’intérêt}
\cvline{\textbf{Informatique}}{Linux (Membre d'une association sur le logiciel libre et la sécurité informatique : iTeam), Réseau, Serveur, Raspberry Pi, Systèmes d'information, VPN}
\cvline{\textbf{Sécurité}}{Sites de challenges (Position : 1295 / 42822), Cryptographie}
%\cvline{Sports}{Snowboard/Ski (Compétition), Wakeboard, Tennis, Krav Maga, Muay Thaï, Volley-ball}
%\cvline{Musiques}{Piano et Guitare durant 8 ans}


\newpage
\section{Expériences Professionnelles}
\cventry{Novembre 2018 - Aujourd'hui}{Spécialiste Cyber Sécurité}{Scorpio Group}{Monaco}{Monaco}{
	\begin{itemize}
		\item \texttt{Sécurité organisationnelle:} Développement de politique de sécurité, de procédures, de standards, etc.
		\item \texttt{Sécurité des infrastructures et des réseaux:} Interne et Cloud,
		\item \texttt{Sécurité des utilisateurs finaux:} Postes de travail, mobiles, etc.
		\item \textit{Audits de sécurité:} Gestion des audits par des prestataires, réalisation d'audits internes,
		\item \textit{Gestion de projets:} Migration dans le cloud, projets de chiffrements, etc.
		\item \textit{Training et Recommandations} Réalisation de communications sur les bonnes pratiques ou l'utilisation d'outils de sécurité.\\
		\item \textbf{\underline{Environnement technique :}}
		\begin{itemize}
			\item \textbf{ISO 27K, Classification de l'information, chiffrement, etc.}
			\item \textbf{Revue des équipements de sécurité de l'infrastructure (Pare-feu, VPN, Certificats, etc.),}
			\item \textbf{Anti-Virus, Anti-Spam}
			\item \textbf{Nessus}\\
		\end{itemize}
	\end{itemize}
}

\cventry{Septembre 2016 - Octobre 2018}{Ingénieur Sécurité / Pentesteur}{Amadeus}{Sophia-Antipolis}{France}{
\begin{itemize}
	\item \texttt{Tests d'intrusion} sur différentes applications (Web App, Web Services, Back-End, Clients lourds),
	\item \texttt{Développement} d'outils afin d'automatiser certaines tâches lors des tests d'intrusion,
	\item \textit{Recherches et études} sur des techniques de sécurisation ou pour des recommendations,
	\item \texttt{Investigations} lors de réponses à incidents,
	\item \texttt{Mise en place} et participation à des projets de sécurité,
	\item \textit{Proof of Concept} concernant de nouvelles technologies de sécurité informatique.\\
	\item \textbf{\underline{Environnement technique :}}
  	\begin{itemize}
  		\item \textbf{Linux, Kalilinux (et ses outils)}
  		\item \textbf{Scanner de vulnérabilités (Nessus, Arachni, OpenVAS, Nikto)}
  		\item \textbf{BurpSuite Pro, Qualys WAS}
  		\item \textbf{Metasploit}\\
  	\end{itemize}
\end{itemize}
}

\cventry{Février 2016 - Juillet 2016}{Stage Ingénieur Sécurité Test d'Intrusion}{\textbf{\textit{AVISTO}}}{Sophia-Antipolis}{France}{
\begin{itemize}
  \item Mise en place d'un \textbf{environnement de pentest} basé sur Kalilinux et \textit{Hackazon},
  \item Génération des logs pour analyse Big Data de détection d'intrusion,
  \item \textbf{Test d'Intrusion} sur le site Web Hackazon et consolidation,
  \item Attaquer et consolider un cluster de calculs Big Data,
  \item Rédaction et publication d'un \textit{livre blanc},
  \item \textbf{Audit de sécurité} d'une partie de l'infrastructure d'AViSTO,
  \item \textit{Présentation commerciale} de la sécurité informatique.\\
  \item \textbf{\underline{Environnement technique :}}
  	\begin{itemize}
  		\item \textbf{Linux, LAMP, Kalilinux (et ses outils)}
  		\item \textbf{Scanner de vulnérabilités (Nessus, Arachni, OpenVAS, Nikto)}
  		\item \textbf{Metasploit}
  		\item \textbf{Raspberry Pi}\\
  	\end{itemize}
\end{itemize}}

\cventry{Mai 2015 - Août 2015}{Stage Data Analyst \& Software Developer}{\textbf{\textit{Amadeus North America}}}{Boston}{USA}{
\begin{itemize}
  \item Analyse d'un important flux de données (Big Data) pour le client KAYAK :
  	\begin{itemize}
  		\item Analyse des feedbacks de Kayak afin de déceler les erreurs concernant les informations envoyées
  		\item Réalisation d'un outil d'automatisation de la recherche d'erreurs :
  			\begin{itemize}
  				\item Intégration d'un outil au sein des différents solutions existantes :
  					\begin{itemize}
  						\item Front-end : HTML5, CSS3, PHP, JS
  						\item Back-end : Python, C++\\
  					\end{itemize}
  			\end{itemize}
  	\end{itemize}
  \item \textbf{\underline{Environnement technique :}}
  	\begin{itemize}
  		\item \textbf{HTML5, CSS3, PHP, JS}
  		\item \textbf{Python, C++}
  		\item \textbf{Programmation Sécurisée}
  		\item \textbf{Base de donnée MySQL}\\
  	\end{itemize}
\end{itemize}}

\cventry{Janvier 2014 - Juillet 2016}{Assistant Professeur TP Informatique}{\textbf{\textit{ECE Paris}}}{Paris}{France}{
\begin{itemize}
  \item Assister le professeur lors des séances de TPs et de TDs.
  \item Répondre aux questions des élèves.
  \item Débloquer les élèves et les assister dans leur rédaction de code informatique.
  \item Aider les élèves à acquérir de bonnes pratiques en terme d'algorithmies et de programmation.
  \item Corriger les examens.\\
\end{itemize}}

\cventry{Juillet 2011 - Aujourd'hui}{Professeur d'informatique}{\textbf{\textit{Freelance}}}{}{}{Cours d’informatique à domicile pour tous niveaux et tous OS, installations, dépannages, etc...
\begin{itemize}
  \item Apprentissage des bases de l'informatique.
  \item Accompagnement des clients dans leurs projets (montage vidéo, etc...).
  \item Conseils et accompagnement lors d'achats de matériel informatique.\\
\end{itemize}}

\cventry{Octobre 2014 - Avril 2015}{Stage/Mineure Projet Personnel Ingénieur Sécurité / Assistant RSSI}{\textbf{\textit{Assemblée-Nationale}}}{Paris}{France}{
\begin{itemize}
  \item Audit de l'infrastructure de l'Assemblée Nationale :
  	\begin{itemize}
  		\item Étude comparative avec la \textbf{PSSIE} de l'ANSSI
  		\item Objectif : \textit{Accréditation} d'accès à une \underline{plateforme diplomatique} sous mention « \textbf{Diffusion Restreinte} »
  		\item Rédaction de bulletin d'information sur les \underline{bonnes pratiques de sécurité informatique}
  		\item Rédaction de guides de \textit{sécurisation des devices IOS et Android}
  		\item Création d'un master Linux pour les machines reconditionnées, et configuration
  		\item Création d'une \textbf{procédure d'effacement sécurisé de données.}\\
  	\end{itemize}
  \item \textbf{\underline{Environnement technique :}}
  	\begin{itemize}
  		\item \textbf{Linux, Windows}
  		\item \textbf{Effacement de données sécurisé : Blancco}
  		\item \textbf{LaTeX}\\
  	\end{itemize}
\end{itemize}}

\cventry{Juin 2014 - Août 2014}{Stage Ingénieur Développement Web et IOS}{\textbf{\textit{SCM Centre d'Imagerie Médicale Belvédère}}}{Nice}{France}{
\begin{itemize}
  \item Rédaction d'un cahier des charges pour obtention de la mission
  \item \textbf{Développement sécurisé} d'un application Web pour la gestion des emplois du temps
  \item Auto-hébergement de la solution et \textbf{intégration sécurisée} au sein du réseau de la société
  \item Création d'une application IOS permettant la consultation des emplois du temps
  \item \textit{Sécurisation} de l'infrastructure et des services
  \item Formation du personnel
  \item Maintenance.\\
  \item \textbf{\underline{Environnement technique :}}
  	\begin{itemize}
  		\item \textbf{HTML5, CSS3, PHP, MySQL}
  		\item \textbf{Objective-C}
  		\item \textbf{Linux, Raspberry, Apache}
  		\item \textbf{Chef de projet, Contact client}
  		\item \textbf{Administration système, sécurité et réseau}\\
  	\end{itemize}
\end{itemize}}

\cventry{Juin 2013 - Juillet 2013}{Stage Ingénieur Sécurité et Pentest}{\textbf{\textit{Schneider-Electric}}}{Carros}{France}{
\begin{itemize}
  \item \textbf{Test de sécurité} sur des automates :
  	\begin{itemize}
  		\item Mise en place d'un \underline{protocole de test de sécurité}
  		\item \textbf{Tests d'intrusion} sur différents modèles d'automates
  		\item Mise en place d'une VM de test
  		\item Création de scripts pour automatiser certains audits
  		\item Rédaction d'un rapport de mission Anglais/Français\\
  	\end{itemize}
  \item \textbf{\underline{Environnement technique :}}
  	\begin{itemize}
  		\item \textbf{Bash, Python, C++}
  		\item \textbf{Linux, Kalilinux (Tests d'intrusion)}
  		\item \textbf{Automates}\\
  	\end{itemize}
\end{itemize}}

\cventry{Avril 2014 - Avril 2015}{Formateur Sécurité et Administrateur Système Sécurité}{\textbf{\textit{iTeam}}}{Paris}{France}{
\begin{itemize}
  \item Association de promotion du \textbf{logiciel libre et de sécurité informatique}
  \item Formateur Sécurité Informatique, Raspberry Pi, et serveurs.\\
  \item \textbf{\underline{Environnement technique :}}
  	\begin{itemize}
  		\item \textbf{Linux, Raspberry Pi}
  		\item \textbf{Serveur Web (Apache, Nginx)}
  		\item \textbf{SSH, FTP, SSL}
  		\item \textbf{Snort, Iptables, Fail2ban}\\
  	\end{itemize}
\end{itemize}}

\cventry{Sept. 2012 - Avril 2014}{Vice-Président Microsoft}{\textbf{\textit{Équipe Système}}}{Paris}{France}{
Association d'informatique de l'ECE Paris.
\begin{itemize}
  \item Poste dirigeant au sein de l'association.
  \item Responsable du pole Microsoft (organisation d'événement en lien avec le poste d'Ambassadeur Microsoft.
  \item Assistance des élèves, en coopération étroite avec le service informatique de l'école, afin de régler les soucis liés à l'infrastructure de l'école (comptes, imprimantes, Wi-Fi, etc...).\\
\end{itemize}}

\cventry{Sept. 2012 - Mai 2013}{Ambassadeur Microsoft}{\textbf{\textit{Microsoft France}}}{Paris}{France}{
\begin{itemize}
  \item Représentant Microsoft au sein de mon Campus.
  \item Formations et démonstrations des derniers produits Microsoft.\\
\end{itemize}}

% \cventry{Nov. 2011 - Décembre 2015}{Administrateur}{\textbf{\textit{Freedom-IP}}}{}{}{Administrateur d’un service de VPN communautaire et de son forum.
% \begin{itemize}
%   \item Gestion d'un forum et de ses utilisateurs.
%   \item Gestion des comptes VPN des utilisateurs.
%   \item Résolutions des problèmes des utilisateurs des services.\\
% \end{itemize}}

%\cventry{Oct. 2013 - Dec. 2014}{Administrateur}{\textbf{\textit{Journal du Pirate}}}{}{}{
%\begin{itemize}
%  \item Rôle de gestion du forum.
%  \item Assistance et conseils envers les membres.
%  \item Entretien du forum.
%\end{itemize}}

%\cventry{Nov. 2011 - Oct. 2013}{Modérateur}{\textbf{\textit{Journal du Pirate}}}{}{}{
%\begin{itemize}
%  \item Rôle de gestion du forum.
%  \item Assistance et conseils envers les membres.
%  \item Entretien du forum.
%\end{itemize}}

%\newpage
\section{Publications}
\cventry{Décembre 2014 - Septembre 2015}{Grenelle du Numérique}{\textbf{\textit{ECE Paris \& CGI}}}{}{}{
\begin{itemize}
  \item Rédaction d'un livre blanc sur le sujet \textit{Big data, ce mot qui va transformer les business models},
  \item En charge de la partie sur le \textbf{Sécurisation du Big Data},
  \item Prise de parole lors de conférence sur le Big Data (Paris Dauphine),
  \item Réalisation d'un spot vidéo informatif,
  \item Faire part de notre expertise aux députés avant le vote des lois sur le numérique de septembre 2015.
  \item \href{http://www.cgi.fr/grenelle-du-numerique/livre-blanc}{http://www.cgi.fr/grenelle-du-numerique/livre-blanc}\\
\end{itemize}}

\cventry{Février 2016 - Juillet 2016}{Stage Pentest AViSTO}{}{}{}{
\begin{itemize}
  \item Rédaction d'un livre blanc sur la \textit{sécurité informatique},
  \item Sensibilisation et formation aux principales failles Web,
  \item Explications sur le fonctionnement des failles, leurs exploitations, leurs corrections, ainsi que le moyen de les détecter (log).
  \item \href{https://fr.scribd.com/document/319529310/Thibault-MILLANT-Pentest-Tests-d-Intrusion-Cas-pratique-sur-le-site-e-commerce-Hackazon}{https://fr.scribd.com/document/319529310/Thibault-MILLANT-Pentest-Tests-d-Intrusion-Cas-pratique-sur-le-site-e-commerce-Hackazon}\\
\end{itemize}}

\end{document}
